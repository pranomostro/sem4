\documentclass{article}

\usepackage{amssymb}
\usepackage{amsmath}
\usepackage{listings}

\begin{document}

\section*{Blatt 1}

\subsection*{T1}

$ \Pr[ABC]=1/3 $ \\*
$ \Pr[ACB]=0 $ \\*
$ \Pr[BAC]=0 $ \\*
$ \Pr[BCA]=1/3 $ \\*
$ \Pr[CAB]=1/3 $ \\*
$ \Pr[CBA]=0 $ \\*

\subsection*{T2}

$$ \Pr[\hbox{Agathe gewinnt}]=\Pr[\hbox{Zahl kommt bei ungeradem Wurf}]=\sum_{i=1}^{\infty} p^{i-1}*(1-p) $$
$$ =(1-p)*\sum_{i=0}^{\infty} p^i $$
$$ =(1-p)*\frac{p}{1-p} $$
$$ =p $$

Anwending der Konvergenz der geometrischen Reihe gegen $\frac{p}{1-p}$ für $|p|<1$.

$$ \Pr[\hbox{Balthasar gewinnt}]=1-\Pr[\hbox{Agathe gewinnt}]=1-p $$

Das Spiel ist genau dann fair, wenn $p=1-p \Leftrightarrow p=0.5$.

\subsection*{T3}

\subsubsection*{(a)}

Die Wahrscheinlichkeit, dass der $i$-te Blaubeermuffin keine Blaubeere
enthält, ist $(\frac{m-1}{m})^n$, also die Wahrscheinlichkeit, dass
für jede Blaubeere ein Muffin gewählt wird, der nicht der $i$-te ist.

\subsubsection*{(b)}

%TODO: Learn how to write the binomial coefficient in LaTeX

$ \Pr[\hbox{Alle echt}] $ \\*
$ = 1-\Pr[\hbox{Mindestens einer nicht echt}] $ \\*
$ = 1-\Pr[\bigcup_{i=0}^{m} \hbox{i nicht echt}] $ \\*
$ = 1-(m*(\frac{m-1}{m})^n-\hbox{m over 2}*(\frac{m-2}{m})^n+\hbox{m over 3}*(\frac{m-3}{m})^n-\dots $ \\*
$ = 1-(\sum_{i=1}^{m}(-1)^{i+1}*\hbox{m over i}*(\frac{m-i}{m}))^n) $

\subsubsection*{(c)}

Using Klong.

\begin{lstlisting}
	.l("nstat")
	ame::{[n m];m::x;n::y;1-+/{((-1)^1+x)*kc(x;m)*((m-x)%m)^n}'1+!m}
	{ame(6;x)<0.9}{x+1}:~0
23
\end{lstlisting}

\subsection*{H1}

\subsubsection*{(a)}

Using Klong.

\begin{lstlisting}
	#flr({(0=x!4)|0=x!9};1+!150)
49
\end{lstlisting}

$\frac{49}{150} \approx 32.67\% $

\subsubsection*{(b)}

Falls die Quersumme größer als 3 ist, kann keine der Ziffern größer
als 3 sein. Damit kommen nur die Zahlen 1, 2, 3, 10, 11, 12, 20, 21,
30, 100, 101, 102, 110, 111 und 120 in Frage, insgesamt 16 Zahlen.

$\frac{16}{150} \approx 10.67\% $

\end{document}
