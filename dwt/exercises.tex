\documentclass{article}

\usepackage{amssymb}
\usepackage{amsmath}
\usepackage{listings}

\begin{document}

\section*{Blatt 1}

\subsection*{T1}

$ \Pr[ABC]=1/3 $ \\*
$ \Pr[ACB]=0 $ \\*
$ \Pr[BAC]=0 $ \\*
$ \Pr[BCA]=1/3 $ \\*
$ \Pr[CAB]=1/3 $ \\*
$ \Pr[CBA]=0 $ \\*

\subsection*{T2}

$$ \Pr[\hbox{Agathe gewinnt}]=\Pr[\hbox{Zahl kommt bei ungeradem Wurf}]=\sum_{i=1}^{\infty} p^{i-1}*(1-p) $$
$$ =(1-p)*\sum_{i=0}^{\infty} p^i $$
$$ =(1-p)*\frac{p}{1-p} $$
$$ =p $$

Anwending der Konvergenz der geometrischen Reihe gegen $\frac{p}{1-p}$ für $|p|<1$.

$$ \Pr[\hbox{Balthasar gewinnt}]=1-\Pr[\hbox{Agathe gewinnt}]=1-p $$

Das Spiel ist genau dann fair, wenn $p=1-p \Leftrightarrow p=0.5$.

\subsection*{T3}

\subsubsection*{(a)}

Die Wahrscheinlichkeit, dass der $i$-te Blaubeermuffin keine Blaubeere
enthält, ist $(\frac{m-1}{m})^n$, also die Wahrscheinlichkeit, dass
für jede Blaubeere ein Muffin gewählt wird, der nicht der $i$-te ist.

\subsubsection*{(b)}

%TODO: Learn how to write the binomial coefficient in LaTeX

$ \Pr[\hbox{Alle echt}] $ \\*
$ = 1-\Pr[\hbox{Mindestens einer nicht echt}] $ \\*
$ = 1-\Pr[\bigcup_{i=0}^{m} \hbox{i nicht echt}] $ \\*
$ = 1-(m*(\frac{m-1}{m})^n-{m \choose 2}*(\frac{m-2}{m})^n+{m \choose3}*(\frac{m-3}{m})^n-\dots $ \\*
$ = 1-(\sum_{i=1}^{m}(-1)^{i+1}*{m \choose i}*(\frac{m-i}{m}))^n) $

\subsubsection*{(c)}

Using Klong.

\begin{lstlisting}
	.l("nstat")
	ame::{[n m];m::x;n::y;1-+/{((-1)^1+x)*kc(x;m)*((m-x)%m)^n}'1+!m}
	{ame(6;x)<0.9}{x+1}:~0
23
\end{lstlisting}

\subsection*{H1}

\subsubsection*{(a)}

Using Klong.

\begin{lstlisting}
	#flr({(0=x!4)|0=x!9};1+!150)
49
\end{lstlisting}

$\frac{49}{150} \approx 32.67\% $

\subsubsection*{(b)}

Falls die Quersumme größer als 3 ist, kann keine der Ziffern größer
als 3 sein. Damit kommen nur die Zahlen 1, 2, 3, 10, 11, 12, 20, 21,
30, 100, 101, 102, 110, 111 und 120 in Frage, insgesamt 16 Zahlen.

$\frac{16}{150} \approx 10.67\% $

\subsection*{H2}

$$\Pr[E_n]=\begin{cases}\frac{2}{3}*\frac{1}{2^n} & \hbox{falls n gerade}\cr
		  \frac{1}{3}*\frac{1}{2^n} & \hbox{falls n ungerade}\end{cases}$$

Begründung: $\frac{2}{3}*\frac{1}{2^n}=2*\frac{1}{3}*\frac{1}{2^n}$,
und $\frac{1}{3}*\frac{1}{2^n}+\frac{2}{3}*\frac{1}{2^n}=\frac{1}{2^n}$.

Da alle Ereignisse disjunkt sind, kann man einfach addieren und sieht:

$$ \Pr[\bigcup_{n \in \mathbb{N}} E_n] = \sum_{n \in \mathbb{N}} \frac{1}{2^n}=1 $$

\subsection*{H3}

$$ \Pr[\hbox{keiner bekommt seine eigene Abgabe}]= $$
$$ 1-\Pr[\hbox{irgendeiner bekommt seine eigene Abgabe}]= $$
$$ 1-\bigcup_{n \in S} \Pr[\hbox{n kriegt seine eigene Abgabe}]= $$
$$ 1-(\sum_{i=1}^n \frac{1}{n}-\sum_{1 \le i_1 \le i_2 \le n} \frac{1}{n} * \frac{1}{n-1} + \sum \dots)= $$
$$ 1-(n*\frac{1}{n}-{n \choose 2}*\frac{(n-2)!}{n!}+{n \choose 3}*\frac{(n-3)!}{n!} \dots)= $$
$$ 1-(\sum_{k=1}^{n} (-1)^k \frac{(n-k)!}{n!} {n \choose k})= $$
$$ 1-(\sum_{k=1}^{n} (-1)^k \frac{(n-k)!}{n!} * \frac{n!}{(n-k)!*k!})= $$
$$ 1-(\sum_{k=1}^{n} (-1)^k \frac{1}{k!})= $$
$$ 1-\sum_{k=1}^{n} \frac{(-1)^k}{k!} $$

Für $n \rightarrow \infty$ gilt $\lim_{n \rightarrow \infty}
1-\sum_{k=1}^{n} \frac{(-1)^k}{k!}=1-e^{-1}=1-\frac{1}{e} $.

Die Wahrscheinlichkeit, dass keiner der unendlich vielen Studierenden
sein eigenes Blatt zurückbekommt, ist also $1-\frac{1}{e}\approx0.632$.

\subsection*{H4}

\subsubsection*{(a)}

Annahme: Wenn hier "in die $n$-te U-Bahn" geschrieben wird, ist damit
gemeint, dass sie in alle vorher nicht einsteigen konnte, man berechnet
also die Wahrscheinlichkeit, dass sie "genau in die $n$-te U-Bahn"
einsteigt.

%TODO: Diese sind nicht wirklich schön gemacht oder durchdacht.

$ \Pr[\hbox{Agathe kann in die $n$-te U-Bahn einsteigen}]=\frac{1}{n*(n+1)} $

\subsubsection*{(b)}

$ \Pr[\hbox{Agathe steigt spätestens in die $n$-te U-Bahn ein}]=$\\*
$ 1-\Pr[\hbox{Agathe steigt in die $n+1$-ste U-bahn ein}]=1-\frac{1}{n+1} $

\section*{Blatt 2}

\subsection*{T1}

\subsubsection*{(a)}

Da ${6 \choose 2}=15$, gibt es 15 Möglichkeiten, die erste Gruppe
zu bilden. Daraufhin kann man aus den übrigen 4 Personen noch ${4 \choose 2}=6$
Gruppen bilden, und dann bleiben noch 2 Leute übrig, die die
letzte Gruppe bilden. Da aber hier die Gruppen durchnummeriert wurden,
das in der gegebenen Aufgabe aber nicht der Fall ist, wird durch $3!$,
als die Anzahl der möglichen Anordnung der Gruppen, geteilt.  Die Anzahl
der Möglichkeiten, 6 Personen in Zweiergruppen einzuteilen, ist also
$\frac{{6 \choose 2}*{4 \choose 2}}{3!}=\frac{15*6}{6}=15$.

\subsubsection*{(b)}

Für die erste Studentin gibt es 3 Möglichkeiten, einem Studenten
zugeordnet zu werden. Für die zweite gibt es dann noch 2 Studenten,
denen sie zugeordnet werden kann, und für die dritte ist es dann fix.

Damit gibt es $3*2=6$ Möglichkeiten, die Lerngruppen gemischt zu machen,
und die Wahrscheinlichkeit, dass eine Lerngruppe gemischt ist, ist
$\frac{6}{15}=0.4$.

\subsubsection*{(c)}

$$\Pr[\hbox{alle gemischt}|\hbox{A\&D}]=$$
$$1-\Pr[\hbox{keine gemischt}|\hbox{A\&D}]=$$
$$1-\frac{1}{\frac{{4 \choose 2}*{2 \choose 2}}{2!}}=$$
$$1-\frac{1}{3}\approx$$
$$0.667$$

\subsubsection*{(d)}

$$\Pr[(A\&D \cup B\&C) \cap \hbox{alle gemischt}]=$$
$$\Pr[A\&D \cap \hbox{alle gemischt}]=$$
$$\Pr[A\&D]*\Pr[\hbox{alle gemischt}|A\&D]=$$
$$\frac{\frac{{4 \choose 2}}{2!}}{15}*\frac{2}{3}=\frac{12}{45}=$$
$$\frac{6}{15}*\frac{2}{3}=$$
$$\Pr[A\&D \cup B\&C]*\Pr[\hbox{alle gemischt}]$$

\subsection*{T2}

\section*{Blatt 6}

\subsection*{T1}

$X=99*X_{b}$

$$\Pr[X \ge (1+\delta)\mu] \le (\frac{e^{\delta}}{(1+\delta)^{1+\delta}})^\mu$$

$$\mathbb{E}(X)=\mu=\frac{1}{4}$$

$(\frac{e^{\delta}}{(1+\delta)^{1+\delta}})^\mu \le \frac{1}{100}$\\*
$\Leftrightarrow (\frac{e^{\delta}}{(1+\delta)^{1+\delta}})^{\frac{1}{4}} \le \frac{1}{100}$\\*
$\Leftrightarrow \frac{e^{\delta}}{(1+\delta)^{1+\delta}} \le \frac{1}{10^8}$\\*
$\Leftrightarrow e^{\delta}*10^8 \le (1+\delta)^{1+\delta}$\\*

Und:\\*

$(\frac{e^{\delta}}{(1+\delta)^{1+\delta}})^\mu \le \frac{1}{100}$\\*
$\Leftrightarrow (\frac{e^{\delta}}{(1+\delta)^{1+\delta}})^{\frac{3}{4}} \le \frac{1}{100}$\\*
$\Leftrightarrow \frac{e^{\delta}}{(1+\delta)^{1+\delta}} \le \frac{1}{10^8}$\\*
$\Leftrightarrow e^{\delta}*10^8 \le (1+\delta)^{1+\delta}$

\end{document}
